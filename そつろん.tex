\documentclass[12pt,a4j,titlepage]{ltjsarticle}
\usepackage{semi}
\usepackage{here}
\usepackage{listings}

% \title{}
% \author{}
% \date{}

\begin{document}

\begin{titlepage}
  \begin{center}
  
    \vspace*{1.5cm}
    
    {\LARGE 2024年度 卒業論文} 
    
    \vspace*{75truept}
    
    {\Huge  海洋生物の学習コンテンツの開発} %論文タイトル

    \vspace{10truept}

    {\Huge } %論文タイトル 長い場合 改行1

    \vspace{10truept}

    {\Huge } %論文タイトル 改行2

    \vspace{85truept}
    
    {\LARGE 指導教員 須田 宇宙 准教授}
    
    \vspace{60truept}
    
    {\LARGE 千葉工業大学 情報ネットワーク学科}
    
    \vspace{15truept}
    
    {\LARGE 須田研究室}
    
    \vspace{70truept}
    
    {\LARGE 2132125 氏名 古川 璃名 } % 氏名は消さない 学生番号 氏名 名前

    \vspace{70truept}
    
  \end{center}
  \begin{flushright}

    {\LARGE 提出日 2025年 月 日}
  
  \end{flushright}
\end{titlepage}

\setcounter{tocdepth}{3}
% 目次の出力
\tableofcontents
% 表目次
\listoftables
% 図目次
\listoffigures
\clearpage

\section{緒言}\label{緒言}
日本は四方を海で囲まれており,昔から豊かな水産資源により発達してきた.
しかし近年日本人の海に対する関心が薄れており,好感度や海洋問題に対する認知度が下がっている.
加えて近年海水温の上昇や乱獲,海洋環境の悪化などの海洋問題が深刻化し,水産資源が減少している.
それにより海洋生物の生態系だけでなく,人間の生活にも大きく影響を及ぼしている.\par
海洋問題に対する関心を高めるには,海洋環境や海洋生物の生態などに興味を持ってもらうことが有用とされている.
現在海洋生物を学ぶことができるコンテンツは図鑑やWeb(ブログや動画コンテンツなど),水族館などがあるが,情報が点在してしまっていることや,水槽と実際の海とではギャップがあり,深さや広さなどを表現するのに限界があることなどが問題点として挙げられる.\par
そこで本研究では海洋問題を考えるきっかけとして,日本の海の特徴や海洋生物の生息しているエリアや水深などの生態を中心に,海洋環境を学べるコンテンツの作成を目的としている.

\clearpage

\section{日本人と海}\label{人と海}

\subsection{日本近海の特徴}
日本は四方を海に囲まれているため,昔から海と深い関わりがある.
日本は周りを太平洋,東シナ海,日本海,オホーツク海の4つの海に囲まれており,4つの海はそれぞれ異なる生態系を築いている.
加えて寒流である親潮とリマン海流と暖流である黒潮と対馬海流の合わせて4つの海流が近海に流れている.
これにより寒流と暖流がぶつかることで潮目と呼ばれる植物プランクトンが豊富な海域ができ,様々な生物が集まる場所が存在する.
加えて海流によって泳いでくる魚が違うため,日本近海では多くの種類の魚を見ることができ,海流に乗って回遊する魚も移動してくるため,同じ場所でも四季によって異なる種類の魚を見ることができる.\par
また日本の海岸の特徴として,大陸棚と呼ばれる浅瀬が広く広がっている場所やリアス海岸と呼ばれる複雑な形の海岸が多く存在することが挙げられる.
大陸棚は太陽光が届きやすく海藻がよく育つためプランクトンが多く存在し魚が集まりやすく豊かな漁場となっている.
リアス海岸の特徴としては水深が深く海中の酸素が多いことや,山からの養分が多く流れてくること,水質や水温が安定していることが挙げられ,豊かな漁場となっているのと同時に,養殖漁業が盛んな場所となっている.\par
これらの要因から日本近海は多くの魚を観測することができ,海洋生物種の14.6%が生息する豊かな海洋環境となっている.
\subsection{海との関わり}
近年の日本人は海は大切であると認識しながらも,海への好感度はさがり海離れが進んでいる\cite{nihonnzaidann 2024}.
加えて肉よりも魚食だったのが2011以降は逆転し,肉類の消費量は魚介類の消費量を上回り魚離れが進んでいる\cite{suisanntyou suji}.
しかし日本では魚介類の消費量は減少傾向にある一方で世界では増加傾向にあり一人当たりの消費量は過去半世紀で2倍になっている.\par
近年は世界的に水産資源の消費量が増えとことによる魚の乱獲や,海面上昇,海洋ゴミによる魚への被害などの海洋問題が深刻化しているが,日本人の海への関心が薄くなっているため,認知度や意識などは低下している.
この状況を脱するためには魚について知ってもらうことが有効だと考える.
日本人は魚を食べることが好きという人が単純に海が好きという人よりも多くまた,海洋生物などのニュースを見たり,水族館などで実際に見ることで海に関心を持ったという人も少なくない.
そのため海により関心を持ってもらうためには海洋生物や海洋環境について知ってもらうことが有効であり重要である.
\subsection{これまでの取り組み}
日本人の海離れ,魚離れが進む中,水族館や政府は様々な取り組みをしてきた.
日本の海に関わる教育は昔から行われていたが,内容は漁師などの職人や専門家などを育成を中心に行われてきた.
しかし,2007年に海洋基本法が制定されて以降,次世代を支える世代への海に関する教育と,国民の海への理解を深める普及活動の充実が推進されている\cite{kaiyoukaihatu}.
内閣府のホームページには各施設の持つ海に関する教育資源や行っている取り組みなどがまとめられている.
加えて2022年からは「ニッポン学びの海プラットフォーム会合」という「全ての市町村で海洋教育が実践されることを目指し、関係府省・関係機関間の連携を強化するため,情報共有や検討を行う.」\cite{manabitoumi}ことを趣旨とした会合が開催され,より多くの人が海への理解を深めるための取り組みが行われている.\par
また国土地理院の「海しる」というサイトでは博物館,水族館,自然体験施設,海洋教育団体の四つの位置がわかりやすく表示されており,詳しくしくない人でも気軽に海について学びに行けるようになっている.\par
水族館には「種の保存」「教育・環境教育」「調査・研究」「レクリーション」という4つの社会的役割があり,来場者には学びと楽しさを提供している.
例えば水槽に実際に滝つぼや渦潮などを作りより自然に近い状態での展示をしたり,解説パネルに駄洒落を盛り込んだり飼育員の愛情深い手書きパネルなど,親しみやすい物するという取り組みがされている.
さらにプロジェクションマッピングなど新しい技術と掛け合わせたり,話題性を創出することで認知を広め,多くの人に興味を持ってもらえるようなコンテンツを提供し,もともと海や魚に関心のない層が来場者を増やす取り組みが多くされている.
\clearpage

\section{日本の海}\label{海}
\subsection{閉鎖性海域}
閉鎖性海域とは、周囲を陸に囲まれた海のことを指す.
このような海は自然に恵まれ気候がおだやかなため,その海域や沿岸地域には多くに人が生活している.
そのため周辺では多くの産業が生まれ,交通が発達しレクリエーションの場としても利用されてきた。
しかし,多くの人の活動によって海の汚染が進み大きな問題が起こっている.
閉鎖性海域は海水が循環しにくくいったん汚染が進むと,元の状態に戻すことが難しい.
日本では代表的なものとして東京湾や瀬戸内海、伊勢湾などがあげられる。
\subsubsection{東京湾}
東京湾は一番狭い場所で幅6kmしかなく,この狭まった場所から北側が内湾,南側が外湾とよばれる\cite{toukyouwan}.
内湾の平均水深は15mと浅くなっているが外湾に向かて徐々に深くなっている.
湾口は水深600m になる東京海底谷が横たわり、湾全体の最大水深は700mになる.
そのため東京湾には浅瀬に住むハゼなどの小魚から,東京海底谷に生息するミツクリザメなどの深海鮫という多様な生物が生息している.
また東京湾の周囲には60本以上の川が流れており多くの生活排水が流れ込んでいる.
そのため湾内には排水に含まれる栄養を摂取しプランクトンが豊富に存在し,加えて外海に比べ穏やかで天然の養殖場のようになっている.
東京湾で獲れる魚は江戸前の魚として昔からよく食べられていた.
例えば東京湾で獲れるマアジは金アジとも呼ばれ,基本的に回遊魚であるアジが沿岸部の同じ場所で生活し,豊富なプランクトンを食べて育つ.
そのため金アジは一般的なマアジよりは小さめだが脂がよく乗っている\cite{toyoumi}.\par
一方で東京湾の湾口が小さく水の出入りが頻繁に起こらないため,汚濁物質が堆積しやすくなっている.
加えて近年の東京湾は周囲が大きな主要な都市に囲まれ,生活排水だけでなく産業排水が多く流れ込む.
そのため水質の汚染が問題となっており,高度成長期以降魚は激減している
\subsubsection{瀬戸内海}
瀬戸内海は温暖な気候で降水量が少なく,浅瀬が多いことが特徴として挙げられる.
また大小さまざまな島が連なる複雑な地形のため,干満の差が激しく潮流が非常に強いことも挙げられ,海底には砂でできた丘陵地帯が形成されている.
この特徴により鳴門海峡周辺では渦潮と呼ばれる自然現象を見ることができる.
また明石海峡周辺の地形が複雑でマダイの住処となる岩礁や餌となる小魚なども多く生息しているため,「明石のタイ」として有名な漁場となっている.\par
瀬戸内海も第二次世界大戦後海洋汚染が進み水質改善がすすめられ,近年は大きく改善してきた\cite{setouti}.
しかし水質改善に伴い瀬戸内海の貧栄養化が進み漁獲量が減少している.
\subsection{深海湾}
相模湾,駿河湾,富山湾の3つの湾は1000m以上の水深をもち,三大深湾と呼ばれる.
深海湾は浅瀬に住む生物の他に深海に生息する魚も見ることができるため,他の湾に比べて多くの種類の魚を見ることができる.
\subsubsection{相模湾}
相模湾は平塚から小田原にかけての海岸には大陸棚がほとんどなく,海岸から1000mの海底まで一気に深くなる.
また相模湾は開放的な湾であり沖合を流れる黒潮の影響を大きく受ける.
そのためアジやカツオ,マグロなどの回遊性魚類が多く来遊し,約1600種の魚種を見ることができる\cite{sagami}.
\subsubsection{駿河湾}
駿河湾は日本で最も深い湾であり水深が2500m にもなる\cite{suruga}.
加えて海底勾配は急であり海岸から約2kmで水深500m に達する地点もある.
そのため陸地から近いところに深層水が存在する.
深層水は植物性プランクトンの光合成がほとんど行われず,浅い海の海水と交じり合わないため,「低温安定性」「清浄性」「高栄養性」などの特性を持っている.
そのため駿河湾の深海にはタカアシガニなどの貴重な生物が生息している.
加えて沿岸部ニには海底湧水の名所があり駿河湾に注ぎ込むため,日本で唯一桜エビが生息する海としても知られている.
\subsubsection{富山湾}
富山湾は相模湾と同様大陸棚が狭く海岸近くから深い海になっており,平均水深は800mである\cite{toyama}.
加えて「藍瓶」 と呼ばれる16もの海底谷があり起伏にとんだ地形であり,若狭湾に次いで大きな湾となっている.
富山湾にも深層水があり常に2℃以下と冷たいため冷水系の魚が多く住んでいる.
また近くに暖流の対馬海流が流れており,富山湾は上が暖流,下は冷たい深層水という二層になっている.
そのため暖流系の魚と冷水系の魚の両方の魚,合わせて約500種類を見ることができ,特にシロエビ,ホタルイカ,ブリの3種類は「富山県のさかな」に認定されている.\par
シロエビは藍瓶がある水深150~300mに生息しており,富山湾のみ水揚げされている.
水揚げ量が少ない貴重な水産資源であるため定期休漁日を制定するなど,資源管理が行われている.\par
またブリは暖流系の魚であり,春に生まれた稚魚は餌を求めて北海道へ北上し,冬に産卵するために再び九州へ南下する.
夏秋を北海道で過ごし,栄養を蓄え冬に南下する途中の脂の良く乗ったブリが多く取れるため,お歳暮やおせち料理などに使用され,古くから富山県の生活にかかわっている.
ブリは出世魚としても有名で成長段階によって呼び名が変わる珍しい魚である.
\subsection{リアス海岸}
アリス海岸はもともと谷だった地形に海水が流れこんでできたため,岸から離れるとすぐに水深が深くなる.
加えて山からの栄養が川から多く流れ込みプランクトンがよく育つ.
また岬や入り江が入り組んだ複雑な地形になっているため,波が高くなりにくく穏やかであるため,砂が混じることなく形のきれいな身が育つ「耳吊り式」の養殖などの漁業が営まれることが多い.
日本では三陸海岸や若狭湾,志摩半島などがあげられる.\par
三陸海岸は宮古市付近より南側はリアス海岸,北側は隆起海岸となっている\cite{sanriku}.
北側の海岸周辺にはウニやアワビなどの岩礁を好む魚介類が多く生息している.
また小袖海岸は海女の北限として知られ,沿岸では素潜り漁をする海女の姿を見ることができる.
三陸沖には寒流の親潮と暖流の黒潮がぶつかり,潮目と呼ばれる世界でも有数の漁場が形成される.
潮目でプランクトンが大量に発生するため小さな魚から大型の海洋生物まで多様な生物が集まる.
また暖流性の魚と寒流性の魚の両方が海流に乗って泳いでくるため,様々な魚が漁獲される.
しかし現在三陸沖での漁獲量が全盛期に比べ減少している.
特にサケの2023年の漁獲量は134tと全盛期のわずか0.2%となっている.
原因の一つとして挙げられているのは海水温の上昇である.
近年世界規模での海水温上昇が続いているが三陸沿岸は特に著しく,去年の平均水温は17.6度で過去30年の平均水温より約4℃高くなっている.
\subsection{干潟をともなう海}
日本には有明海や八代海,沖縄や三河湾などの海域に多く干潟が存在し全国合わせて約5万4千haもの面積にもなる\cite{higata}.
干潟には1日に2回の潮汐の干満があり,多くが河口付近に広がっている.
そのことにより川からは豊かな栄養が流れ込み,海からはプランクトンが供給されるため,二枚貝などの底生性小動物のよい餌場となり,豊かな生態系が築かれている.
また干潟には水が出入りする際に海水中に大量の酸素を供給し、バクテリアを活性化させ有機物の分解を促進させる効果があり,海を浄化する役割も担っている.
加えて干潟には多くのアサリやハマグリ,マテガイなどの貝類が生息し,潮干狩りなどのレクリエーションの場にもなっており,生物の棲家というだけでなく様々な重要な役割を担っている.
しかし,高度成長期に沿岸域の開発が盛んに行われ,1945年から1998年の約50年で4割もの干潟が減少した.
現在は人口干潟がつくられるなど干潟を再生する試みが行われており,徐々に面積も増加し保全や生態系の回復が進められている.\par
有明海は干満の差が日本一の海であり最大で6.8mの干満差を記録したことがある\cite{ariake}.
また流入河川が多く,大小100を超える河川が流入しているためプランクトンが豊富で,生き物がよく育つ豊かな海になっている.
加えて日本最大の干潟があることも特徴として挙げられる.
これらの特徴により有明海には国内ではほかの海では見ることができない,有明海特産種が23種生息しており,独自の生体系が築かれている.
有明海特産種のうちの1種であるムツゴロウは有明海を代表的な魚で,日本では有明海と八代海の一部のみ生息する珍しい魚である.
潮が引くと巣穴から出てきて大きな胸鰭で歩くようにはい回り,干潟の表面にある珪藻などをなめるようにして食べる魚で,近年埋め立てや環境汚染の影響で数が減少し、絶滅危惧種に指定されているが,保全活動により生息数は安定しているため,現在でも漁が続けられており,食べることができる。
\subsection{サンゴ礁のある海}
サンゴは25℃〜30℃ほどの高水温で水深の浅い綺麗な海域が繁殖に適しており,千葉県の海にも造礁サンゴが生息している.
しかし,サンゴ礁と呼ばれる地形が形成される海域は一般的には鹿児島県の大隅諸島より南方の暖かい海域となっている.\par
沖縄の海は川などが少なく,山の養分が海に流れてこないため海中に浮遊するプランクトンが少ない\cite{okinawa}.
そのため透明度が高くサンゴ礁が発達しやすい環境になっている.
サンゴ礁は様々な魚の住処となっており,その魚を食べる大型の魚も集まるため豊かな生態系が築かれている.
また沖縄県は日本で唯一,熱帯から亜熱帯性海域に点在する島々から構成されているため,とても暖かい海であることが特徴として挙げられる.
そのため温暖な海を好む回遊魚の漁場が形成されている.\par
しかし近年沖縄のサンゴは数を減らしており,大きな要因として生活排水などによる水質汚染や水温上昇などがあげられる.
サンゴが減少してしまうと,これらの働きが失われてしまい数多くの生態系が影響を受け,最終的には大規模な絶滅にも繋がってしまう.
近年は対策としてサンゴの苗を養殖,移植するという活動が行われ数が回復してきている.
\subsection{海氷の海}
日本で海氷を見ることができるのはオホーツク海のみであり,冬季に結氷する海としては世界で最も南に位置する海域として知られていて,時期になると海の大半が海氷に覆われている.
またこの海域は北海道北東部に位置する海で北海道と千島列島,樺太そして大陸に囲まれた,半閉鎖性の海域となっており,世界的に有数の漁場となっている\cite{ohotuku}.
その理由として暖流の宗谷海流と寒流の東樺太海流が合流し,そこで発生するプランクトンの数が多いことや,海底の起伏が大きく魚が育つための栄養分が豊富にあることなどがあげられる.\par
オホーツク海は春頃になると水温の上昇により流氷は融けていく.
約1ヶ月ほどかけて流氷から融け出した水は,豊富な栄養を含み沿岸生物の繁殖に役立っている.
加えて海底が砂礫の海域となっていてホタテガイの好漁場であるため,道内の主生産地となっており,この海域でのホタテガイの水揚げ量は北海道全体の 6 割以上にもなっている .
オホーツク海への海水の流入は主に日本海からの宗谷海峡を通じた暖流系水が多いが,タラやサケなどの寒流系の魚の漁獲量が多い.
また普段は北極や南極に生息しているクリオネが流氷の時期になると,流氷の穴に入り込んで北海道まで来るため,ダイビングなどで見ることができる.
\clearpage

\section{海と魚}\label{海と魚}
\subsection{回遊魚}
回遊魚とは海の中を泳ぎ回って暮らしている魚の総称で,適した水温や餌,産卵場を求めて移動しており,その行動範囲や経路には一定の法則があると言われている\cite{kairyuu}.
代表的な魚として広範囲の海域を回遊するマグロやマアジ,タチウオなどがあげられる.
マグロやカツオのパブリックイメージのせいで回遊魚は止まると死んでしまうという勘違いをしている人が多いが,それらはエラブタを動かすことが出いないため,泳いでエラに酸素を取り込んでいるからであり,マアジなどのエラを動かせる回遊魚は泳ぐことををやめても死ぬことはない.\par
マグロは1年の半分は餌を求めて,もう半分は産卵のために回遊することが多い.
例えばクロマグロは南西諸島周辺や日本海付近が主な産卵場所されており、産卵できるようになるまで3~5年かかるとされている。
孵化後は日本沿岸を餌を求めて回遊するが,1歳ごろになると太平洋を横断しアメリカ大陸で数年過ごし産卵のために日本に戻ってくる,というライフサイクルになっている。\par
またウナギやサケなど河川と海を往復する種も回遊魚である.
しかしウナギもサケも回遊ルートの明確な解明には至っておらず,いまだに不明な点が多い。
例えばシロサケは日本を出た後,オホーツク海から太平洋,ベーリング海を経てアラスカ湾で12月〜5月頃まで過ごす.
その後夏になるとまたベーリング海に戻り12月頃からまたアラスカ湾で過ごす,という回遊ルートとされている.
そして卵から数えて4年ほどたった9月から12月に日本の川に帰るのが一般的とされている.
\subsection{寒流}
日本近海に流れ込んでいる寒流は主に親潮とリマン海流の2種類であり,寒流には暖流と比べ餌となるプランクトンが多く存在することが特徴として挙げられる\cite{kairyuu}.
一般的にプランクトンの成長に必要な栄養分は水温の低い海底に多く存在する.
しかし寒流は海底付近と水温の差があまりないため撹拌され,通常は海底にある栄養分が太陽光の届くところまで湧き上がる.
そのため植物プランクトンが繁殖し,海中の栄養分が豊富になる.\par
寒流系の海域に生息する魚類は寒流性魚類と呼ばれ,年間を通じて12〜13℃前後またはそれよりも低い場所で生活している.
代表的なものとしてタラやホッケ,サンマ,サケなどの魚が多く漁獲されている.\par
サンマは秋の味覚とも言われ2024年には約2.6万t漁獲されているが,2010年以降漁獲量は減少しており,大きな要因として温暖化などの気候変動が挙げられている\cite{huryou}.
サンマは1年のうちに約1000kmもの距離を成長しながら回遊し秋頃になると日本近海へと近づいてくる.
しかしサンマの回遊ルートは不明な点が多く,様々な場所で稚魚が観測されることから,いつどこで生まれたサンマがどのような回遊ルートを通って日本近海へやってくるのか,解明されていない.
サンマは浮魚という海の表層を泳ぐ魚に分類され,青魚と呼ばれるマグロやサバと同じ種類とされている.
青魚の体色は背中が濃い青,腹は銀白という色合いの種が多くなっている.
これは表層を泳いで生活しているため上から鳥に,下から大型に魚類に狙われることが多く,その際に見つかりにくくするためである\cite{iro}.
またサンマは胃がなく,消化器官が一直線になっている.
同じ青魚であり回遊魚でもあるサバやイワシとは大きく異なった生態となっている.
\subsection{暖流}
日本近海には主に黒潮と対馬海流の2つの暖流が流れ込んでいる.
暖流は寒流よりも透明度が高いという特徴があり,その1番の要因としてプランクトンの量の差が挙げられる.
寒流が濁って見えるのはプランクトンが光をさえぎるためであり,親潮は緑や茶色に見えることがある.
一方で暖流である黒潮は透明度が高く,光の透過などによって美しい青色や黒色に見える.\par
暖流系の海域に生息する魚類は暖流性魚類と呼ばれ,16〜17℃以上の水温で正常に育つとされている.
代表的なものとしてカツオやマグロ,ブリ,イワシ,珊瑚礁域に生息するコーラルフィシュなどの魚が挙げられる.\par
マイワシは暖流性の魚だが分布域が広く,日本では広範囲で漁獲され2024年の漁獲量は約68万tにもなり,日本で一番水揚げされている魚である.
本州から九州の周辺で産卵したイワシは暖流に乗って北上し,春から夏にかけて10°C〜17°Cの海を求め北海道付近を回遊する.
その後秋ごろの水温の低下に伴い南下し,本州の沖合に移動する.
マイワシは水温が6°C以下になると仮死状態になるため,季節の変わり目などに水温が急激に低下すると大量に死んでしまうということもある.\par
また,カツオはサバ,マイワシに次いで漁獲量が多く,旬が2回あるといわれている.
カツオは日本の南方で生まれた後に黒潮に乗って餌を求めて北上し,北海道付近で秋ごろまで過ごす.
その後餌をたくさん食べたカツオが越冬や,産卵のために再び南下する.
春ごろに北上するカツオは初ガツオと呼ばれさっぱりとした味わいで,秋ごろに南下するカツオは戻りガツオと呼ばれ脂ののった味わいに変化する.\par
コーラルフッシュと呼ばれる珊瑚礁に生息している魚の多くは鮮やかな体色をしている\cite{nettaigyo}.
これにはカラフルな珊瑚礁の中で隠れやすいことや,警戒色として天敵から身を守っている,紫外線から身を守っている,など様々な説があるがはっきりしたことはわかっていない.
代表的なものにチョウチョウウオやクマノミが挙げられ,性転換をする種が多く存在する.
\subsection{低魚}
低魚とは海洋の下層または海底で生活している種の総称で,体が扁平もしくは丸いものが多く,典型的な魚とは少し違う独特な生態を持つ魚が多い.
代表的なものとして海底に着床して生活するカレイ目やアンコウ類,穴子類,泥底近くを遊泳するタチウオなどが挙げられる.
体色は赤色や黒色が多く,浮魚のような青色をしているものは少ない.\par
また遊泳速度は遅いため多くの種類は回遊せず,その代わりに小魚や貝類などの餌を捉える瞬発力に優れた速筋を多く有している.
この特徴は根つき魚と呼ばれる1つの海域にとどまっているタイやフグ,タラなどにも当てはまり,筋肉が白いため一般的に白身魚と呼ばれている\cite{iro}.
しかしヒラメやマダイなどは産卵や越冬のために長距離回遊を行うことがわかっている.
一方で回遊魚の多くは長距離を泳ぐのに適した遅筋を多く有しているため赤身魚とも呼ばれ,その中でも青魚はドコサヘキサエン酸などの脂質量やが多いため栄養価が高いとされている.

\clearpage

\section{水産資源と海洋問題}\label{水産資源と海洋問題}

\subsection{水産資源とは}
水産資源とは海洋や河川などの水域で漁業や養殖などの対象となる魚介類や海藻類などの資源を指している.
水産資源は再生可能な資源といわれており,適切に管理すれば永続的な利用が可能となっている\cite{suisanntyou}.
そのため種苗の放流や漁場環境の整備などの資源を増す取り組みや,乱獲や海洋環境の悪化を防止し資源を保護する取り組み,そして海洋環境の現状を知ってもらいことが重要である.
\subsection{漁業の発達と現在}
日本は四方を海に囲まれており豊富な水産資源によって発展してきた.
2.1で述べた通り日本近海は様々な特徴があり,豊かな漁場に囲まれている.
そのため日本の第2次世界大戦以降の水産業は沿岸漁業から沖合,遠洋漁業へと漁場を拡大することで発展し,ピーク時の1984年には生産量は1282万tにもなっていた.\par
しかし平成以降漁業は減少傾向にあり,領海や国連海洋法条約の制定などにより遠洋漁業は全体の1割ほどとなり,沿岸漁業も沿岸の開発による水産生物の減少や環境の変化から生産量が減少している.
現在海面での漁業で生産量が沖合漁業に次いで養殖業が2番目に多く,内水面では漁業よりも養殖業の方が資産量が多くなっている.\par
また近年にはTotal Allowable Catch(以下TAC)制度という制度を制定している.
これは漁獲量を管理することで毎年一定の産卵親魚を残し,再生可能な資源状態を保つことを目的としており,魚だけでなく未来の漁業を守るためにも重要な取り組みとなっている\cite{TAC}.
TAC制度は「漁獲量が多く国民生活上重要な魚種」「資源環境が悪く緊急に管理を行うべき魚種」「日本周辺で外国人により漁獲されている魚種」の3つの条件のいづれかに該当し,かつ科学的知見の蓄積がある7種類の魚種が対象となっている制度で,今後はブリなども加えられる予定である.\par
\subsection{近年減少した海洋生物とその遠因}
近年減少したといわれているのがニホンウナギである.
主な理由は過剰な漁獲と環境の変化の2つであるといわれており,それぞれの要因に対して適切な対応を行う必要がある\cite{nihonunagi}.
しかしニホンウナギの減少の要因の大きさを比較することは難しく,また減少要因に関する知見は不足している.
そのため予防原則と順応的管理の考え方に沿って私たちは環境に対してできることをするしかない.\par
また沖縄県の暖かい海で豊かに見られてサンゴ礁が白骨化してしまったのも大きな話題となった.
2011年に発行された「Reefs at Risk」という調査報告によれば,世界のサンゴ礁の58%が海洋汚染や生物資源の乱獲などの潜在的人間の活動によって脅かされているとされている\cite{sango}.
特に1997年から1998年にかけての世界的な海水温の上昇によって、それまでに例をみない大規模な白化現象が起こった.
サンゴ礁は小さな魚のすみかや餌場になっており,サンゴ礁が減ることでそこに住んでいた魚にも大きな影響が出ている.
\clearpage


\section{開発したコンテンツについて}\label{開発したコンテンツ}
\subsection{海洋生物についての学習}
海洋生物について学ぶ手段の中で主要なものとして水族館が挙げられる.
水族館は魚が海中でどのように生活し,生態系を築いているのかなどをわかりやすく展示しており,海洋生物だけでなく海洋問題の教育に取り組んでいる水族館も多く,水族館に行ったことで環境問題に関心を持ったという人は少なくない.
水族館は海まで行かなくても実際に泳いでいる魚を様々な角度から見ることができるほぼ唯一の場所である.
そのため魚の姿や色,泳ぎ方などを学ぶためには水族館で泳いでいる魚を見ることが有効である.
また1.3で述べた通り展示方法が従来より自然に近いものが増えており,魚と自然の力強さなどの魅力もより学ぶことができるようになっている.\par
水族館の他には図鑑やWebなども挙げられる.
近年は面白い雑学的な情報や意外性のある情報を中心に載せることで,掲載されている生き物に興味がなくても買ってしまうという人が多くなっている.
加えて水族館まで行くのは大変という人でも現代の日本では,動画や知識人の解説などからある程度の知識を得ることができるという考えも多い.\par
しかし水族館では実際の海を再現する事には限界があり,生息している水深や位置,同じ場所でも時期によって見れる魚が違うということを水槽で表現することが難しいという問題点がある.
また書籍は読んだとしても生息している地域や海域がどんなところなのかまで開設しているものは少なく,また生物中心の解説なのでどの魚が同じ海で生活しているのかなどが読むだけでは想像することが難しいという問題点がある.
海洋環境と絡めて学ぶにはあまり適していないだろう.
\subsection{開発目的}
本コンテンツは利用者が海への関心をより高めるためのきっかけとして,海洋生物が生活している海や水深を学び,近年低下している海洋問題への意識向けてもらうことを目的としている.
本コンテンツでの学習できる画面は,図鑑や水族館などを見るだけではわかりづらい魚の生息水深を,スクロールというなじみ深い方法で主体的に体感できる海中画面と,魚の生息域に大きく関係する海流などの海洋環境についても学べるMAP画面の2つで構成されている.
\clearpage

\subsection{オープニング画面}
図\ref{fig:op}1番最初の画面でコンテンツ名が表示される.
海洋と魚を知ってほしいという思いを込めて「カイヨウ魚知(かいよううぉっち)」と名付けた.
スタートボタンを押すと MAP 画面に移動する.
\begin{figure}[h]
\begin{center}
\includegraphics[width=100mm]{op.png}
\end{center}
 \caption{op画面}
 \label{fig:op}
\end{figure}

\subsection{MAP画面}
この画面では日本近海の簡単な海洋環境について学ぶことができる.
図\ref{fig:MAP1}はスタートを押したすぐ後の画面になっている.
左側の水色のボタンで表示するコンテンツを選ぶことができ,海流と地形の2つから選ぶことができる.
図\ref{fig:MAP2}海流のボタンを選択した際の画面で,海流名にカーソルを合わせることで,それぞれの回遊する魚や特徴などの簡単な情報を読むことができる.
またもう一度同じボタンを押すことで初めのMAP画面に戻すことができる.
日本地図の上に表示されている黄色い魚型のボタンを選択することで,海中画面に移動することができる.
\begin{figure}[h]
\centering
\begin{minipage}[b]{0.49\columnwidth}
\centering
\includegraphics[width=75mm]{MAP1.png}
 \caption{MAP画面}
 \label{fig:MAP1}
\end{minipage}
\begin{minipage}[b]{0.49\columnwidth}
\centering
\includegraphics[width=75mm]{MAP2.png}
 \caption{MAP海流画面}
 \label{fig:MAP2}
 \end{minipage}
\end{figure}


\subsection{海中画面}
この画面では魚がどのくらいの水深で生活しているのかを見ることができ,MAP画面で黄色い魚のいずれかを選択することでこの画面に移動する.
選択できる海は,オホーツク海,三陸海岸,東京湾,相模湾,駿河湾,富山湾,瀬戸内海,有明海,沖縄の海の9つを用意している.
どの海も地形や海洋環境に特徴があり,それぞれが個性的な生態系築いている.
ここでは東京湾の海中画面を用いてコンテンツの説明をしていく.
図\ref{fig:sea1}東京湾付近のボタンを押すと移動する.
最初に表示される魚の種類は決まっているが,場所はある程度ランダムになっていて,リロードするたびに変化する.
画面構成は左上に海の名前,その下には画面内の水深,横にはその時に表示される魚の名前が表示されるようになっている.
図\ref{fig:sea2}のようにカーソルを合わせることで表示できる.
また海の名前の横には戻るボタンが表示されており,押すとMAP画面に戻る.
画面をスクロールすることで水深は増減し,背景の色と魚の種類が変化する.
表示される魚は実際の生息水深と対応しており,東京湾のような浅瀬が広い海では水深が浅い場所は魚を多く表示し,深い場所に表示する魚は少なくしている.

\begin{figure}[h]
\centering
\begin{minipage}[b]{0.49\columnwidth}
\centering
\includegraphics[width=75mm]{sea1.png}
 \caption{東京湾海中画面1}
 \label{fig:sea1}
\end{minipage}
\begin{minipage}[b]{0.49\columnwidth}
\centering
\includegraphics[width=75mm]{sea2.png}
 \caption{東京湾海中画面2}
 \label{fig:sea2}
 \end{minipage}
\end{figure}

\begin{figure}[h]
\centering
\begin{minipage}[b]{0.49\columnwidth}
\centering
\includegraphics[width=75mm]{sea3.png}
 \caption{東京湾海中画面3}
 \label{fig:sea3}
\end{minipage}
\begin{minipage}[b]{0.49\columnwidth}
\centering
\includegraphics[width=75mm]{sea4.png}
 \caption{東京湾海中画面4}
 \label{fig:sea4}
 \end{minipage}
\end{figure}

\section{コンテツの実装}\label{コンテンツの実装}
\subsection{データ構造}
本コンテンツは場所の名前を持ったデータベースと,海ごとに魚の名前と水深,表示する数などの詳細な情報を管理しているデータベースの2つが主なデータとなっっており,その2つからデータを持ってきて処理することで海中画面に表示している.
このことにより1つ1つの海で表示画面と魚のデータを管理するよりも,簡単に管理することができるようになっている.
海中画面は1つのejsコードで管理できるようになっており,また新しく選択できる海を増やす際や,表示する魚を増やす際のデータの追加が簡単にできるようになっている.
場所と魚のデータ項目をそれぞれを表\ref{table:place}と表\ref{table:fish-data}に示す.
\begin{table}[h]
\centering
  \caption{場所データの項目}
  \label{table:place}
  \small
  \begin{tabular}{rr}
  name &  名前  \\
   type &  fish-dataと同じ名前  \\
  \end{tabular}
  \end{table}

\begin{table}[h]
\centering
  \caption{魚の詳細なデータの項目}
  \label{table:fish-data}
  \small
  \begin{tabular}{rr}
   type &  画像の名前\\
   name & 魚の名前 \\
   depthm &  表示する水深の上限 \\
   depthM &  表示する水深の下限\\
   number & 表示する画像の数 \\
   z & 最前面から何番目に表示するか  \\
   width & 画像の大きさ \\
  \end{tabular}
  \end{table}
\subsection{パララックス効果}
パララックスとは直訳すると視差という意味で,Webデザインにおいて立体感や遠近感を持たせることができる表現方法を指している.
例えば通常ユーザーがWebページをスクロールすると背景やコンテンツは一緒に上下に動く.
これにパララックス効果を利用することで,背景画像は前景のコンテンツよりも遅い速度で移動させることができる.
このことによりWebサイトに奥行きや動きを感じられるようすることができる.\par
パララックス効果を利用することで従来の静的Webページよりもユーザーの興味を惹きつけることができるという利点がある.
従来のWebページよりも時系列の変化などを背景の明暗や,前景のコンテンツの動きや変化などで視覚的に強調して見せることができる.
パララックスはユーザーがページをスクロールすることで情報が表示されるため,場面や時代の移り変わりを実感させることに適している.
そのため物語の体験や歴史の紹介をする際などに利用するとより効果的である.
一方で視覚的な効果が多すぎると重要な情報を見落としてしまうリスクが増えてしまうため,慣れていないユーザーが見ると混乱を招いてしまうという欠点もある.\par
本コンテンツでは海の奥行きを表現するために使用しており,スクロールした際に前面にある画像と奥にある画像の上下の移動距離を変えている.
これにより立体感が増し,より水深の深い場所へ移動することを強調しコンテンツに動きをつけることで,使用者がより視覚的に楽しめるようになっている.
\clearpage

\section{結言}\label{結言}
日本は四方を海で囲まれており,昔から豊かな水産資源により発達してきたが,近年日本人の海に対する関心が薄れており,海洋問題に対する認知度が下がっている.
加えて近年海水温の上昇や乱獲,海洋環境の悪化などの海洋問題が深刻化し,水産資源が減少したことで,生態系や人間の生活に大きく影響を及ぼしている.\par
海洋問題に対する関心を高めるには,海洋環境や海洋生物の生態などに興味を持ってもらうことが有用とされている.
そこで本研究では海洋問題を考えるきっかけとして,日本の海の特徴や海洋生物の生息しているエリアや水深などの生態を中心に,海洋環境を学べるコンテンツの作成した.
本コンテンツは図鑑や水族館で見ているだけでは分かりづらい水深を体感できるコンテンツになっており,海洋生物の生態を学ぶ上では他のコンテンツとの差別化を行った.
利用者が主体的に画面を動かすコンテンツになっており,加えて日本近海の海洋環境についても併せて学ぶことができるため,海洋生物が生活する海への解像度がより上がるだろう.
しかし本件ん急で取り上げた海は9つと少なく,また海洋生物も限られてしまっている.
今後は世界の海の中を見れるようにするなど改善の余地があると考える.\par
本コンテンツを利用することで,海洋生物や海洋環境に興味を持ち,そこから近年低下してしまった海への好感度や海洋問題への認知度の回復につながることを望む.
\clearpage

\section{謝辞}
本研究の遂行及び本論文の作成にあたり,須田研究室の仲間に多くの手助けを頂きました,深く感謝の意を表します.そして,本論文の作成にあたり多大なる御指導及び御助言を頂きました,須田宇宙准教授に深く感謝の意を表します.
\clearpage

\begin{thebibliography}{99}
\bibitem{nihonnzaidann 2024}  日本財団: "「海と日本人」に関する意識調査 2024" , \url{https://www.nippon-foundation.or.jp/wp-content/uploads/2024/07/new_inf_20240711_01.pdf},2024/12/25参照
\bibitem{manabitoumi} 内閣府:"ニッポン学びの海プラットフォーム会合の開催について",\url{https://www8.cao.go.jp/ocean/policies/education/manabi/pdf/03/settigami.pdf},2024/1/10参照
\bibitem{suisanntyou suji}  水産庁: "令和4年度水産白書" , \url{https://www.jfa.maff.go.jp/j/kikaku/wpaper/r04_h/trend/index.html},2024/12/25参照
\bibitem{TAC}  水産庁: "TACを知る‼︎" , \url{https://www.jfa.maff.go.jp/j/koho/pr/pamph/pdf/tac001.pdf},2024/12/25参照
\bibitem{kaiyoukaihatu}  海洋開発をめぐる諸相:総合調査書,2024/12/25参照
\bibitem{suisanntyou}  水産庁: "海の中の状況、水産資源について" , \url{https://www.jfa.maff.go.jp/j/kikaku/wpaper/R4/LP/2.html},2024/12/25参照
\bibitem{toukyouwan}  環境省: "東京湾水環境サイト" , \url{https://water-pub.env.go.jp/water-pub/mizu-site/mizu/wotb/summary1.html},2025/1/10参照
\bibitem{toyoumi}  豊海おさかなミュージアム: "解説ノート" , \url{https://museum.suisan-shinkou.or.jp/},2024/12/25参照
\bibitem{setouti}  環境省: "せとうちネット" , \url{https://www.env.go.jp/water/heisa/heisa_net/setouchiNet/seto/g1/g1chapter1/index.html},2024/12/25参照
\bibitem{sagami}  神奈川県: "水産業" , \url{https://www.pref.kanagawa.jp/menu/5/19/94/index.html},2025/1/10参照
\bibitem{suruga}  静岡県: "駿河湾" , \url{https://www.pref.shizuoka.jp/machizukuri/kowan/1040846/1025741.html},2025/1/10参照
\bibitem{toyama}  富山魚連: "富山湾の漁業" , \url{https://www.toyama-sakana.com/fisheries},2025/1/10参照
\bibitem{sanriku}  環境省: "多種多様性の観点んから重要度の高い海域" , \url{https://www.env.go.jp/nature/biodic/kaiyo-hozen/kaiiki/engan/11701.html},2025/1/10参照
\bibitem{ariake}  福岡有明海漁業協同組合連合会: "福岡有明海の魅力" , \url{https://fukuoka-ariake.com/attraction/sealife/},2025/1/10参照
\bibitem{higata}  水産庁: "沿岸域の環境・生態系保全活動の進め方(暫定指針)" , \url{https://www.jfa.maff.go.jp/j/kikaku/tamenteki/sankou/attach/pdf/index-4.pdf},2025/1/10参照
\bibitem{okinawa}  沖縄県: "沖縄の海" , \url{https://www.pref.okinawa.jp/kyoiku/kodomo/1002681/1002683/index.html},2025/1/10参照
\bibitem{ohotuku}  北海道立総合研究機構: "水産研究本部" , \url{https://www.hro.or.jp/fisheries/h3mfcd0000000gsj/marine/o7u1kr000000019q/o7u1kr000000dgp2/o7u1kr000000dgps.html},2025/1/10参照
\bibitem{nihonunagi}  水産庁: "ウナギをおめぐる状況と対策について" , \url{https://www.jfa.maff.go.jp/j/saibai/pdf/meguru.pdf},2025/1/10参照
\bibitem{sango}  ローレッタ・バーク、ケイティ・レイター、マーク・スポルディング、アリソン・ペリー: "Reefs at Risk" (2011), \url{https://www.jfa.maff.go.jp/j/saibai/pdf/meguru.pdf},2025/1/10参照
\bibitem{nettaigyo}  大島範子: "魚の体色と変化"  ,色材協会誌,2016 年 89 巻 6 号 p. 178-183,2025/1/10参照
\bibitem{huryou}  水産庁: "不良問題に対する検討会とりまとめについて" , \url{https://www.jfa.maff.go.jp/j/study/attach/pdf/furyou_kenntokai-21.pdf},2025/1/10参照
\bibitem{iro} 角謙二.知っておきたい魚の基本新装版
\bibitem{} 野村祐三.旬の美味い魚を知る本
\bibitem{} 野村祐三.地魚大全
\bibitem{kairyuu} 鈴木香里武.水も世界の秘密がわかる!凄すぎる海の生物図鑑
\end{thebibliography}
\end{document}