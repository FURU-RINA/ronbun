\documentclass[twocolumn,a4j,10pt]{ltjsarticle}
\usepackage{kougai}

\title{海洋生物の学習コンテンツの開発}
\author{2132125 古川璃名  指導教員 須田 宇宙 准教授}
\date{}

\begin{document}

\maketitle

\section{はじめに}
\label{sec:description1}
日本は四方を海で囲まれ豊かな水産資源により発達してきた.
しかし近年海に対する関心が薄れてきている\cite{関心}ことに加え,海水温の上昇や乱獲などの海洋問題が深刻化した影響で水産資源が減少し,人間の生活や海洋生物の生態系に大きく影響を及ぼしている.\par
海洋問題に対する関心を高めるには,海洋環境や海洋生物の生態などに興味を持ってもらうことが有用である\cite{海洋環境}.
現在海洋生物を学ぶことができるコンテンツは図鑑やWeb(ブログや動画コンテンツなど),水族館などがあるが,情報が点在してしまっていることや,水槽と実際の海とではギャップがあり,深さや広さなどを表現するのに限界があることなどが問題点として挙げられる.\par
そこで本研究では海洋問題を考えるきっかけとして,日本の海の特徴や海洋生物の生息しているエリアや水深などの生態を中心に,海洋環境を学べるコンテンツの作成を目的としている.

\section{豊富な水産資源}
\label{sec:description2}
日本の周りには4つの海洋がありそれぞれが異なる生態系を築いている.
そこから海流によって様々な生物が運ばれることにより日本近海では様々な魚を見ることが出来る.
加えて大陸棚と呼ばれる太陽光が届きやすく海藻がよく育つ場所にはプランクトンが多く存在し,様々な生物が集まっている.
これらの要因から日本近海には海洋生物種の14.6%が生息する豊かな海洋環境となっている.\par
しかし近年海洋汚染や地球温暖化、乱獲などの影響で海洋生物が減少している.
これ以上水産資源を失わないためにも海洋生物や環境に興味を持ってもらうことが重要だと考える.

\section{コンテンツ内容}
\label{sec:description3}
本コンテンツは海に関心のない人が使用した際に海洋生物や海洋環境に興味を持ってもらえるよう,解説文などの文字を少なくしている.
加えて使用者が楽しめるよう多種多様な生物が表示され,使用者自身が画面をスクロールすることで表示される魚や位置が簡単に変えることができ,関心のない人でも飽きずに利用できるようになっている.
この画面にはパララックス効果を使用しており,奥行きを表現し魚に動きを持たせることで,利用者により興味をもって学習してもらえるようになっている.\par
本コンテンツでは日本近海の海洋や海の位置関係を学べる画面と,様々な場所の海での魚の生息する水深について学べる画面を用意した.
メインとなる魚の生息する水深について学べる画面では水族館では表現できない,どの魚がどのくらいの水深で生活しているのかがわかりやすいよう,利用者が主体的にスクロールすることで,魚同士の位置関係を体感できるようになっている.\par
図\ref{fig:東京}は東京湾に生息している魚を見ることが出来る画面になっている.
本コンテンツでは各海ごとに10〜15匹の魚を見ることがでいる.
東京湾の内湾は平均水深が15mと浅く,湾口に近づくにつれ深くなり最大水深は700mになる.
そのため東京湾では浅瀬に住む小魚から深海に住む魚という多様な生物を見ることが出来る.
この特徴を画面で表現するために水深の浅い場所には魚を多く表示させ,深くなるにつれ表示する魚を少なくしている.
浅い場所の魚を多く表示することで,東京湾には浅瀬が広いことを表現することができ,また漁獲される魚も浅瀬の魚が多いことが表現できる.\par
浅い場所にはハゼやアジ,スズキなどの東京湾を代表する江戸前魚が多く生息しており,コンテンツに反映している.
深い場所ではミツクリザメなどを見ることができる.
ゴンズイなどの群れで生活している魚はまとめて表示できるようにしている.
左側には選択した海の名前,スクロール量に応じた水深,生息している魚の名前が表示される.
画面を下にスクロールすることで左上の水深が深くなり魚の種類も変化していく.
\begin{figure}[h]
\begin{center}
 \includegraphics[clip,width=85mm,height=55mm]{toukyouphot.png}
\end{center}
 \caption{東京湾海中}
 \label{fig:東京}
\end{figure}




\section{おわりに}
\label{sec:description3}
このコンテンツをきっかけに海洋問題について考え,行動する人が増えることを期待する.

\begin{thebibliography}{99}
\bibitem{関心}日本財団 “「海と日本人」に関する意識調査 2024”
\bibitem{海洋環境}環境省海洋生物多様性保全攻略公式サイト,\url{https://www.env.go.jp/nature/biodic/kaiyo-hozen/step/step06.html}
\end{thebibliography}
\end{document}